% Options for packages loaded elsewhere
\PassOptionsToPackage{unicode}{hyperref}
\PassOptionsToPackage{hyphens}{url}
%
\documentclass[
]{book}
\usepackage{amsmath,amssymb}
\usepackage{lmodern}
\usepackage{ifxetex,ifluatex}
\ifnum 0\ifxetex 1\fi\ifluatex 1\fi=0 % if pdftex
  \usepackage[T1]{fontenc}
  \usepackage[utf8]{inputenc}
  \usepackage{textcomp} % provide euro and other symbols
\else % if luatex or xetex
  \usepackage{unicode-math}
  \defaultfontfeatures{Scale=MatchLowercase}
  \defaultfontfeatures[\rmfamily]{Ligatures=TeX,Scale=1}
\fi
% Use upquote if available, for straight quotes in verbatim environments
\IfFileExists{upquote.sty}{\usepackage{upquote}}{}
\IfFileExists{microtype.sty}{% use microtype if available
  \usepackage[]{microtype}
  \UseMicrotypeSet[protrusion]{basicmath} % disable protrusion for tt fonts
}{}
\makeatletter
\@ifundefined{KOMAClassName}{% if non-KOMA class
  \IfFileExists{parskip.sty}{%
    \usepackage{parskip}
  }{% else
    \setlength{\parindent}{0pt}
    \setlength{\parskip}{6pt plus 2pt minus 1pt}}
}{% if KOMA class
  \KOMAoptions{parskip=half}}
\makeatother
\usepackage{xcolor}
\IfFileExists{xurl.sty}{\usepackage{xurl}}{} % add URL line breaks if available
\IfFileExists{bookmark.sty}{\usepackage{bookmark}}{\usepackage{hyperref}}
\hypersetup{
  pdftitle={Inference in multivariate autoregressive process and its extensions},
  pdfauthor={Pierre Gloaguen},
  hidelinks,
  pdfcreator={LaTeX via pandoc}}
\urlstyle{same} % disable monospaced font for URLs
\usepackage{longtable,booktabs,array}
\usepackage{calc} % for calculating minipage widths
% Correct order of tables after \paragraph or \subparagraph
\usepackage{etoolbox}
\makeatletter
\patchcmd\longtable{\par}{\if@noskipsec\mbox{}\fi\par}{}{}
\makeatother
% Allow footnotes in longtable head/foot
\IfFileExists{footnotehyper.sty}{\usepackage{footnotehyper}}{\usepackage{footnote}}
\makesavenoteenv{longtable}
\usepackage{graphicx}
\makeatletter
\def\maxwidth{\ifdim\Gin@nat@width>\linewidth\linewidth\else\Gin@nat@width\fi}
\def\maxheight{\ifdim\Gin@nat@height>\textheight\textheight\else\Gin@nat@height\fi}
\makeatother
% Scale images if necessary, so that they will not overflow the page
% margins by default, and it is still possible to overwrite the defaults
% using explicit options in \includegraphics[width, height, ...]{}
\setkeys{Gin}{width=\maxwidth,height=\maxheight,keepaspectratio}
% Set default figure placement to htbp
\makeatletter
\def\fps@figure{htbp}
\makeatother
\setlength{\emergencystretch}{3em} % prevent overfull lines
\providecommand{\tightlist}{%
  \setlength{\itemsep}{0pt}\setlength{\parskip}{0pt}}
\setcounter{secnumdepth}{5}
\usepackage{booktabs}

%%%%%%%%%% COMMANDS %%%%%%%%%%%%%%%%%%ù
\newcommand{\rmd}{\text{d}}
\newcommand{\Cov}{\mathbb{C}\text{ov}}
\newcommand{\e}{\text{e}}
\newcommand{\inv}{^{-1}}
\newcommand{\GoI}{\left(\Gamma \otimes I \right)}
\newcommand{\IoG}{\left(I \otimes \Gamma \right)}
\newcommand{\GpG}{\left( \Gamma \oplus \Gamma \right)}
\newcommand{\bs}[1]{\boldsymbol{#1}}
\newcommand{\Vect}[1]{\textbf{vec}\left(#1 \right)}
\newcommand{\Sinf}{\mathbf{S}}
\newcommand{\M}{\mathbf{M}}

\ifluatex
  \usepackage{selnolig}  % disable illegal ligatures
\fi
\usepackage[]{natbib}
\bibliographystyle{apalike}

\title{Inference in multivariate autoregressive process and its extensions}
\author{Pierre Gloaguen}
\date{2022-01-22}

\begin{document}
\maketitle

{
\setcounter{tocdepth}{1}
\tableofcontents
}
\hypertarget{about-these-notes}{%
\chapter*{About these notes}\label{about-these-notes}}
\addcontentsline{toc}{chapter}{About these notes}

This works gathers some notes about maximum likelihood inference in probabilistic model involving the well known Gaussian autoregressive process.

The main motivation for this work was the inference aroung discretely observed integrated Ornstein Uhlenbeck process in movement ecology.

The book starts by describing this movement model in ecology and the different configurations relative to it. Then, succesive chapters build the different models by adding different bricks.
The objective is to show how the inference can be performed for each model.

\hypertarget{integrated-ornstein-uhlenbeck-process}{%
\chapter{Integrated Ornstein-Uhlenbeck process}\label{integrated-ornstein-uhlenbeck-process}}

\hypertarget{notations}{%
\section{Notations}\label{notations}}

In the following:

\begin{itemize}
\tightlist
\item
  \(I\) is the identity matrix whose dimension should be explicit in the concerned expression.
\item
  \(\otimes\) is the Kronecker product;
\item
  For square matrices \(A\) and \(B\), \(A \oplus B = A\otimes I + I \otimes B\) is the Kronecker sum;
\item
  For a matrix \(A\), \(\Vect A\) is the stack operator, i.e.~the vectorization of \(A\) by stacking its columns.
  \textbackslash end\{itemize\}
\end{itemize}

\hypertarget{definition-of-the-integrated-ornstein-uhlenbeck-process}{%
\section{Definition of the Integrated Ornstein Uhlenbeck process}\label{definition-of-the-integrated-ornstein-uhlenbeck-process}}

We consider here a stochastic process \(\left\lbrace V_t, X_t \right\rbrace_{t \geq 0}\) taking values in \(\mathbb{R}^{d}\times \mathbb{R}^{d}\) (typically, \(d = 2\) in movement ecology), following an integrated Ornstein Uhlenbeck process.
Formally, \(\left\lbrace V_t \right\rbrace_{t \geq 0}\) is supposed to be the solution of a linear stochastic differential equation, and \(\left\lbrace V_t, \right\rbrace_{t \geq 0}\) is its integration over time.
This can be write in the following way:
\begin{equation}
\begin{array}{rl}
\rmd V_t &= -\Gamma \left( V_t - \mu \right)\rmd t + \Sigma \rmd W_t,~~V_0 = v_0\\
X_t &= x_0 + \int_0^t V_s \rmd s
\end{array}
\label{eq:IOU-SDE}
\end{equation}
where \(\mu\) is vector of \(\mathbb{R}^d\), \(\Gamma\) is a \(d \times d\) invertible matrix, and \(\Sigma\) is a \(d \times d\) matrix such that \(\Sigma\Sigma^T\) is symmetric and positive definite.

This process is a Gaussian Markov process, such that for a positive time \(t\).
\[\begin{pmatrix}
V_{t + \Delta}\\
X_{t + \Delta}
\end{pmatrix} \vert \begin{pmatrix}
V_{t} \\
X_{t}
\end{pmatrix} \sim \mathcal{N}_{2d}\left( \begin{pmatrix}
m^V(\Delta)\\
m^X(\Delta)
\end{pmatrix}, \begin{pmatrix}
\Sigma(\Delta) = \begin{pmatrix}
C^{V}(\Delta) & \left(C^{X, V}(\Delta)\right)^T\\
C^{X, V}(\Delta) & C^{X}(\Delta)
\end{pmatrix}
\end{pmatrix} \right), \]
where:
\begin{align*}
m^V(\Delta) &= \mu + \e^{-\Gamma \Delta} (V_t - \mu)\\
m^X(\Delta) &= X_t + I\mu \Delta + \Gamma\inv(I -  \e^{-\Gamma \Delta})(V_t - \mu)\\
 C^V(\Delta) &= \Sinf -\e^{-\Gamma \Delta}\Sinf \e^{-\Gamma^T \Delta}\\
 C^{X,V}(\Delta) &= \Sinf \Gamma^{-T}\left(I - \e^{-\Gamma^T \Delta}\right) + \Gamma\inv\left(\e^{-\Gamma t} - I\right)\Sinf  \e^{-\Gamma^T t}\\
 C^X(\Delta) &= \M \Delta - \left(I - \e^{-\Gamma^T \Delta}\right)\Gamma\inv \M - \M\Gamma^{-T}\left(I - \e^{-\Gamma \Delta}\right)+ \Gamma\inv \Sinf \Gamma^{-T} -
   \e^{-\Gamma \Delta}\Gamma\inv\Sinf\Gamma^{-T}\e^{-\Gamma^T \Delta},
\end{align*}
where \(\Sinf\) is the matrix such that:
\begin{equation*}
\Vect{\Sinf} = \GpG\inv\Vect{\Sigma \Sigma^T},
\end{equation*}
and \(\M\) is the matrix:
\begin{equation*}
\M= \Sinf\Gamma^{-T} + \Gamma\inv\Sinf.
\end{equation*}

\hypertarget{link-with-the-autoregressive-process}{%
\section{Link with the autoregressive process}\label{link-with-the-autoregressive-process}}

If we denote \(Y_t = \begin{pmatrix} V_{t} \\ X_{t} \end{pmatrix}\), one can see that:
\begin{equation*}
Y_{t + \Delta}\vert Y_t = m(\Delta) + \mathbf{A}(\Delta)  Y_t + E_t,~E_t \sim\mathcal{N}(0,\Sigma(\Delta))
\end{equation*}
where
\[
m(\Delta) =
\begin{pmatrix}
(I - \e^{-\Gamma\Delta}) \mu\\
(\Delta I - \Gamma\inv + \Gamma\inv\e^{-\Gamma\Delta})\mu
\end{pmatrix},
\]
and
\[\mathbf{A}\left(\Delta\right) = \begin{pmatrix}\e^{-\Gamma\Delta} & \mathbf{0} \\
\Gamma\inv - \Gamma\inv\e^{-\Gamma\Delta} &I_2\end{pmatrix}.\]

One can also see that we have:
\[Y_{t + \Delta}\vert Y_t - \overbrace{\begin{pmatrix}\mathbf{0} \\
I_2\end{pmatrix}}^{:= \mathbf{A}_X}X_t = m(\Delta) +  \overbrace{\begin{pmatrix}\e^{-\Gamma\Delta}\\
\Gamma\inv - \Gamma\inv\e^{-\Gamma\Delta}\end{pmatrix}}^{:=\mathbf{A}_V(\Delta)} V_t + E_t\]

Therefore, if we have a sample \(Y_0,\dots, Y_n\), recorded at times \(0,\Delta,\dots, n\Delta\)

\[\tilde{\mathbf{Y}} = \mathbf{X}\mathbf{B} + \mathbf{E}\]
où
\[\tilde{\mathbf{Y}} = \begin{pmatrix}Y_{1} -\mathbf{A}_X X_0 \\
\vdots \\ 
Y_{n} - -\mathbf{A}_X X_{n - 1}
\end{pmatrix}\]
\[\mathbf{X} = \begin{pmatrix}
1 & V_0'\\
\vdots & \vdots \\
1 & V'_{n -1}
\end{pmatrix}\]
\[\mathbf{B} = \begin{pmatrix}
m(\Delta)'\\
A'_V(\Delta)
\end{pmatrix}\]

\hypertarget{simpleAR}{%
\chapter{Multivariate autoregressive model}\label{simpleAR}}

In this chapter, we focus on the case where observations consist in a multivariate time series \(y_0, \dots, y_n\) such that for any \(0\leq t \leq n\), \(y_t \in \mathbb{R}^d\), we denote:
\[y_t = 
\begin{pmatrix}
y_{t,1}\\
\vdots\\
y_{t, d}
\end{pmatrix}\]

\hypertarget{model}{%
\section{Model}\label{model}}

We assume that these observations are realisations of random variables \(Y_0,\dots, Y_n\) such that:
\begin{align}
Y_0 &\sim \chi_0(\cdot),\nonumber \\
Y_t &= m + \mathbf{A}Y_{t -1} + E_t,~1\leq t \leq n \label{eq:AR-simple}
\end{align}
where \(\chi_0\) is some probability distribution over \(\mathbb{R}^d\), \(m\in\mathbb{R}^d\) and \(\mathbf{A}\in \mathcal{M}_{d\times d}\) are parameters, \(E_t\) is a \(d-\)dimensionnal vector such that:
\begin{equation*}
E_t \overset{ind.}{\sim} \mathcal{N}\left(0, \mathbf{\Sigma}\right).
\end{equation*}

\hypertarget{inference}{%
\section{Inference}\label{inference}}

In this simple context, inference consists in finding the maximum likelihood estimates of unknown parameters\footnote{In the case where multiple time series are observed, unknown parameters for the initial distribution $\chi_0(\cdot)$ could be considered} \(\hat{m}\), \(\hat{\mathbf{A}}\) and \(\mathbf{\Sigma}\).

Inference is straightforward here as we can recognize in \eqref{eq:AR-simple} a multivariate linear model:
\[\mathbf{Y} = \mathbf{XB} + \mathbf{E},\]
where
\begin{align*}
\mathbf{Y} &=
\begin{pmatrix}
Y_1'\\
\vdots\\
Y_n'
\end{pmatrix} \in \mathcal{M}_{n \times d},\\
\mathbf{X} &=
\begin{pmatrix}
1 & Y_0'\\
\vdots\\
1 & Y_{n-1}'
\end{pmatrix} \in \mathcal{M}_{n \times (d+1)},\\
\mathbf{B} &=
\begin{pmatrix}
m'\\
A'
\end{pmatrix} \in \mathcal{M}_{(d + 1) \times d},\\
\mathbf{E} &=
\begin{pmatrix}
E_1'\\
\vdots\\
E_n'
\end{pmatrix} \in \mathcal{M}_{n \times d}.
\end{align*}
Thus, \(\hat{m}\) and \(\hat{\mathbf{A}}\) can be obtained using the classical estimate
\begin{equation}
\hat{\mathbf{B}} = \left(\mathbf{X'X}\right)^{-1}\mathbf{X}'\mathbf{Y}, \label{eq:AR-simple-B-hat}
\end{equation}
and \(\hat{\mathbf{\Sigma}}\) is then obtained classicaly as:
\begin{equation}
\hat{\mathbf{\Sigma}} = \frac{1}{n} \sum_{t = 1}^n \left(Y_t - \hat{m} - \hat{\mathbf{A}} Y_{t - 1}\right) \left(Y_t - \hat{m} - \hat{\mathbf{A}} Y_{t - 1}\right)' \label{eq:AR-simple-Sigma-hat}
\end{equation}

\hypertarget{switching-autoregressive-system}{%
\chapter{Switching autoregressive system}\label{switching-autoregressive-system}}

In this chapter, we focus on a more complex system involving autoregressive structure.
We still focus on a time series \(y_0, \dots, y_n\) of values in \(\mathbb{R}^d\). However, it is now supposed that the time series dynamics could change through time, according to an unobserved stochastic process in a discrete space.
This unobserved process might model different regimes of the dynamics (see \citet{rabiner1989tutorial} for selected applications).

\hypertarget{model-1}{%
\section{Model}\label{model-1}}

Taking the same notations and dimensions as in equation \eqref{eq:AR-simple}
We assume that these observations are realisations of random variables \(Y_0,\dots, Y_n\) such that:
\begin{align*}
Z_0 &\sim \chi_{0, Z}(\cdot)\\
Z_t \vert Z_{t - 1} &\sim p(z_t \vert Z_{t - 1}) \\
Y_0 &\sim \chi_{0, Y}(\cdot \vert Z_0), \\
Y_t \vert Z_t &= m(Z_t) + \mathbf{A}(Z_t)Y_{t -1} + E_t,~1\leq t \leq n
\end{align*}
where

\begin{itemize}
\tightlist
\item
  \(\left\lbrace Z_t \right\rbrace_{0\leq t \leq n}\) is an homogeneous Markov chain taking value on the finite space \(\mathbb{K} = \lbrace1,\dots, K \rbrace\), and of transition matrix denoted by \(\mathbf{P}\),
\item
  \(\lbrace m(k)\in\mathbb{R}^d,~\mathbf{A}(k)\in \mathcal{M}_{d\times d}\rbrace_{k = 1,\dots, K}\) are unknown parameters.
\item
  \(\chi_{0, Z}(\cdot)\) and \(\chi_{0, Y}(\cdot \vert Z_0)\) are some probability distributions over \(\mathbb{K}\) and \(\mathbb{R}^d\)
\item
  \(p(z_t\vert Z_{t-1})\) is the law of \(Z_t\) condtionnally to \(Z_{t - 1}\) (here the line of \(\mathbf{P}\) given by \(Z_{t - 1}\)).
\item
  \(E_t\) is a random vector such that:
  \begin{equation*}
  E_t \overset{ind.}{\sim} \mathcal{N}\left(0, \mathbf{\Sigma}\right).
  \end{equation*} where \(\mathbf{\Sigma}\) is \(d\times d\) covariance matrix.
\end{itemize}

In this context, the set of unknown parameters is given by
\[\theta = \left\lbrace \mathbf{P}, m(k),~\mathbf{A}(k), \mathbf{\Sigma}\right \rbrace_{k = 1,\dots, K}.\]

\hypertarget{inference-1}{%
\section{Inference}\label{inference-1}}

In this context, the inference task is twofold:

\begin{enumerate}
\def\labelenumi{\arabic{enumi}.}
\tightlist
\item
  Obtaining maximum likelihood estimate of \(\theta\);
\item
  Retracing the hidden sequence \(Z_0,\dots Z_n\) given the observations \(Y_{0:n}\).
\end{enumerate}

It is well known that these two tasks are indeed complementary.
A common way to solve these problems is the Expectation Maximization (EM) algorithm
\citep{dempster1977maximum}.
The algorithm is shortly depicted here.

\hypertarget{comment-about-notations}{%
\subsection*{Comment about notations}\label{comment-about-notations}}
\addcontentsline{toc}{subsection}{Comment about notations}

In the following, we use the notation \(p\) for a generic probability distribution. The law to which it refers is explicit through arguments. For instance \(p(y_{0:n} \vert z_{0:n})\) is the p.d.f. of a Gaussian vector, the random vector \(Y_{0:n}\vert Z_{0:n}\), and \(p(z_t\vert z_{t - 1})\) is the law of the discrete random variable \(Z_{t} \vert Z_{t - 1}\) evaluated at \(z_t\) and \(z_{t -1}\).
In this context, \(p(z_t\vert z_{t - 1}) = \mathbb{P}(Z_t = z_t \vert \lbrace Z_{t-1} = z_{t -1}\rbrace.\)

\hypertarget{likelihood}{%
\subsection{Likelihood}\label{likelihood}}

A straightforward way to compute the likelihood in this model can be obtained, just using the Markov properties of this model:
\begin{align}
L(\theta \vert y_{0:n}) &:= p(y_{0:n} \vert \theta) \nonumber \\
&= \sum_{z_{0:n}} p(y_{0:n}, z_{0:n}) \nonumber \\
&= \sum_{z_{0:n}} p(z_{0:n})p(y_{0:n}\vert z_{0:n}) \nonumber \\
&=  \sum_{z_{0:n}} p(z_0 \vert \theta) p(y_0\vert z_0, \theta) \prod_{t = 1}^n p(z_{t} \vert z_{t - 1}, \theta)p(y_{t}\vert y_{t -1}, z_{t}, \theta)  \label{eq:AR-HMM-likelihood}.
\end{align}
For a known \(\theta\), every term in \eqref{eq:AR-HMM-likelihood} can be computed.
However, in a general setting, there exists \(K^{n + 1}\) possible sequences \(z_{0:n}\), which makes this direct computation hardly feasible for any common values of \(n\).

\hypertarget{complete-log-likelihood}{%
\subsection{Complete log-likelihood}\label{complete-log-likelihood}}

In the context of missing data problems, an important function is the \emph{complete log-likelihood}, i.e.~the likelihood of the \emph{completed} observations (what we wish we could observe), \((y_{0:n}, z_{0:n})\).
We have that:

\begin{align}
\ell(\theta \vert y_{0:n}, z_{0:n}) :=& \log p(y_{0:n}, z_{0:n} \vert \theta) \nonumber \\
=& \log p(y_{0:n}, z_{0:n}) \nonumber \\
=& \log p(z_{0:n}) + \log p(y_{0:n}\vert z_{0:n}) \nonumber \\
=&  \log p(z_0 \vert \theta) + \log p(y_0\vert z_0, \theta) \nonumber \\
&+ \sum_{t = 1}^n \log p(z_{t} \vert z_{t - 1}, \theta) + \sum_{t = 1}^n \log p(y_{t}\vert y_{t -1}, z_{t}, \theta)  \label{eq:AR-HMM-complete-log-likelihood}.
\end{align}

We emphasize here that, if the hidden states were known, this complete log-likelihood can could easily be computed.

\hypertarget{expectation-maximization-algorithm}{%
\subsection{Expectation-Maximization algorithm}\label{expectation-maximization-algorithm}}

A workaround to find the maximum likelihood estimate in this context is the EM algorithm.
For a given set of parameters, say \(\theta^{(0)}\), let's consider the following function of \(\theta\):

\begin{equation}
Q(\theta \vert \theta^{(0)}) := \mathbb{E}[\ell(\theta \vert Y_{0:n}, Z_{0:n}) \vert Y_{0:n} = y_{0:n}, \theta^{(0)}] \label{eq:E-step-function}.
\end{equation}
A first interesting property of this function is that, for a parameter, say, \(\theta^{(1)}\):
\[Q(\theta^{(1)} \vert \theta^{(0)}) \geq Q(\theta^{(0)} \vert \theta^{(0)}) \Rightarrow L(\theta^{(1)} \vert y_{0:n}) \geq L(\theta^{(0)} \vert y_{0:n}).\]
This property naturally suggests an algorithm to obtain the MLE in this model:

\begin{itemize}
\tightlist
\item
  \textbf{Initialization:} Given a set of observations \(y_{0:n}\), set initial guess \(\theta^{(0)}\)
\item
  \textbf{Iteration:} For \(k\) in \(1,\dots, n\):

  \begin{enumerate}
  \def\labelenumi{\arabic{enumi}.}
  \tightlist
  \item
    \textbf{E step}: Compute \(Q(\theta \vert \theta^{(k-1)})\);
  \item
    \textbf{M step}: Set \(\theta^{(k)} = \text{argmax}_\theta Q(\theta \vert \theta^{(k-1)})\).
  \end{enumerate}
\end{itemize}

It can be shown that the resulting sequence \(\lbrace\theta^{(k)} \rbrace_{k \in \mathbb{N}}\) has a non decreasing set of likelihood values, and it converges towards a critical point (i.e., where the gradient of the likelihood is 0).

In this section, one can ask what's the meaning of \emph{computing} \(Q(\theta \vert \theta^{(k-1)})\).
As this function of \(\theta\) is an expectation with respect to a discrete distribution (the distribution of missing values conditionnally to the observations, under the parameter \(\theta^{(k-1)}\)), \emph{computing} this function means to compute the related weights, that depends on the the observations and on \(\theta^{(k-1)}\).

Indeed, we have:

\begin{align}
Q(\theta \vert \theta^{(0)}) :=& \mathbb{E}[\ell(\theta \vert Y_{0:n}, Z_{0:n}) \vert Y_{0:n} = y_{0:n}, \theta^{(0)}]\nonumber \\
=& \sum_{z_{0:n}} \ell(\theta \vert y_{0:n}, z_{0:n}) p(z_{0:n} \vert y_{0:n}, \theta^{(0)}) \text{d} z_{0:n} \nonumber \\
=& \sum_{k = 1}^K p(z_0 = k \vert y_{0:n}, \theta^{(0)})\left(\log p(z_0 = k \vert \theta) + \log(p(y_0 \vert z_0 = k, \theta))\right)  \nonumber \\
& + \sum_{k = 1}^K\sum_{k' = 1}^K \sum_{t = 1}^n p(z_{t-1} = k, z_{t} = k'\vert y_{0:n}, \theta^{(0)}) \log p(z_{t} = k' \vert z_{t - 1} = k,\theta) \nonumber \\
& + \sum_{k = 1}^K \sum_{t = 1}^n p(z_t = k\vert y_{0:n}, \theta^{(0)}) \log p(y_{t} \vert z_{t} = k, \theta) \nonumber
\end{align}

\hypertarget{e-step-the-baum-welch-smoothing-algorithm}{%
\subsection{E step, the Baum-Welch smoothing algorithm}\label{e-step-the-baum-welch-smoothing-algorithm}}

The E step requires to compute \(p(z_t = k \vert y_{0:n}, \theta^{(0)})\) and
\(p(z_{t - 1} = k, z_{t} = k' \vert y_{0:n}, \theta^{(0)})\).

This is done in an twofold iterative way using the fact that:

\begin{align*}
p(z_t \vert y_{0:n}, \theta^{(0)}) &\propto
\left\lbrace
\begin{array}{ll}
p(y_{0:t}, z_t \vert{\theta^{(0)}})p(y_{(t+1):n} \vert z_t, y_t, \theta^{(0)})  &~0\leq t < n, \\
p(y_{0:n}, z_n \vert{\theta^{(0)}})&~t = n.
\end{array}
\right. \\
p(z_{t - 1}, z_t \vert y_{0:n}, \theta^{(0)}) &\propto
\left\lbrace
\begin{array}{ll}
p(y_{0:{t - 1}}, z_{t -1} \vert{\theta^{(0)}}) p(z_{t} \vert z_{t -1}, \theta^{(0)}) p(y_t \vert z_t, \theta^{(0)}) p(y_{(t + 1):n} \vert z_{t }, y_t, \theta^{(0)}),  &~1\leq t < n, \\
p(y_{0:{n - 1}}, z_{n -1} \vert{\theta^{(0)}}) p(z_{n} \vert z_{n -1}, \theta^{(0)})  p(y_n \vert z_n, y_{n -1}, \theta^{(0)}), &~t = n.
\end{array}
\right.
\end{align*}

The key point here is the appearance of two key quantities, \(\alpha_t(k) := p(y_{0:{t}}, z_{t} = k \vert{\theta^{(0)}})\) and \(\beta_{t}(k) := p(y_{(t+1):n}\vert z_{t} = k, \theta^{(0)})\). To compute these quantities, one can use the following recursions:

\begin{itemize}
\tightlist
\item
  \textbf{Forward recursion}:

  \begin{itemize}
  \tightlist
  \item
    \emph{Initialization}: \[\alpha_0(k) = p(z_0 \vert \theta^{(0)}) p(y_0\vert z_0, \theta^{(0)})\]
  \item
    \emph{Induction}: \[\alpha_t(k) = p(y_t\vert z_t, \theta^{(0)}) \sum_{j = 1}^K \alpha_{t - 1}(j) p(z_t = k\vert z_{t -1} = j, \theta^{(0)})\]
  \end{itemize}
\item
  \textbf{Backward recursion}:

  \begin{itemize}
  \tightlist
  \item
    \emph{Initialization}: \[\beta_n(k) = 1\]
  \item
    \emph{Induction}: \[\beta_t(k) = \sum_{j = 1}^K p(z_{t + 1} = j \vert z_t = k,\theta^{(0)}) p(y_{t+1} \vert z_{t + 1} = j, \theta^{(0)}) \beta_{t+1}(j).\]
  \end{itemize}
\end{itemize}

It thus remains to perform the M-step.

\hypertarget{m-step-constrained-optimization-and-weighted-least-square}{%
\subsection{M step, constrained optimization and weighted least square}\label{m-step-constrained-optimization-and-weighted-least-square}}

The goal is now to maximize, with respect to \(\theta\), the function:
\begin{align*}
Q(\theta \vert \theta^{(0)})&= \sum_{k = 1}^K p(z_0 = k \vert y_{0:n}, \theta^{(0)})\left(\log p(z_0 = k \vert \theta) + \log(p(y_0 \vert z_0 = k, \theta))\right)  \nonumber \\
& + \sum_{k = 1}^K\sum_{k' = 1}^K \sum_{t = 1}^n p(z_{t-1} = k, z_{t} = k'\vert y_{0:n}, \theta^{(0)}) \log p(z_{t} = k' \vert z_{t - 1} = k,\theta) \nonumber \\
& + \sum_{k = 1}^K \sum_{t = 1}^n p(z_t = k\vert y_{0:n}, \theta^{(0)}) \log p(y_{t} \vert z_{t} = k, \theta), \nonumber
\end{align*}
where all terms that do not depend on \(\theta\) are known.

Here, again, we skip the estimation of the terms of \(\theta\) related to initial distributions \(\chi_{0,Z}\) and \(\chi_{0, X}\), which can only be obtained when multiple trajectories (and then initial points) are observed.

We thus focus on updating the transition matrix \(\mathbf{P}_{ij},~1\leq i, j \leq K\) and dynamics parameters \(\lbrace m(k), \mathbf{A}(k), \mathbf{\Sigma}(k)\rbrace_{k = 1,\dots, K}\).

\hypertarget{updating-mathbfp}{%
\subsection*{\texorpdfstring{Updating \(\mathbf{P}\)}{Updating \textbackslash mathbf\{P\}}}\label{updating-mathbfp}}
\addcontentsline{toc}{subsection}{Updating \(\mathbf{P}\)}

We recall that \(\mathbf{P} = \left \lbrace\mathbf{P}_{kk'} := p(z_{t} = k' \vert z_{t - 1} = k, \theta) \right\rbrace_{k,k' = 1,\dots, K}\). \(\mathbf{P}\) is a stochastic matrix, and then satisfies the \(K\) constraints:
\(g(\mathbf{P}_{k\bullet}): = \sum_{k'=1}^K P_{kk'} = 1, 1\leq k \leq K\).
Thus, maximizing \(Q(\theta \vert \theta^{(0)})\) with respect to \(\mathbf{P}\) only requires to focus on the terms
\[\sum_{k' = 1}^K \sum_{t = 1}^ np(z_{t-1} = k, z_{t} = k'\vert y_{0:n}, \theta^{(0)}) \log \mathbf{P}_{kk'} := f(\mathbf{P}_{k\bullet}).\]
Using the Lagrange multiplier method, we want to solve the \(K\) systems of \(K + 1\) equations:
\begin{align*}
\nabla f(\mathbf{P}_{k\bullet}) &= \lambda_k \nabla g(\mathbf{P}_{k\bullet})\\
g(\mathbf{P}_{k\bullet}) &= 1,
\end{align*}
where \(\nabla\) is the gradient operator, and \(\lambda_k \in \mathbb{R}^*\) is called the Lagrange multiplier. In this case, we therefore have:
\begin{align*}
\frac{\sum_{t = 1}^n p(z_{t-1} = k, z_{t} = k'\vert y_{0:n}, \theta^{(0)})}{\mathbf{P}_{k,k'}} &= \lambda_k \\
\sum_{k' = 1}^K{\mathbf{P}_{kk'}} &= 1.
\end{align*}
This leads directly to:
\begin{align*}
\lambda_k &= \sum_{t = 1}^n p(z_{t - 1} = k \vert y_{0:n}, \theta^{(0)}) \\
\mathbf{P}_{kk'} &= \frac{\sum_{t = 1}^n p(z_{t - 1} = k, z_{t} = k'\vert y_{0:n}, \theta^{(0)}) }{\sum_{t = 1}^n p(z_{t - 1} = k \vert y_{0:n}, \theta^{(0)})}
\end{align*}

\hypertarget{updating-dynamics-parameters}{%
\subsection*{Updating dynamics parameters}\label{updating-dynamics-parameters}}
\addcontentsline{toc}{subsection}{Updating dynamics parameters}

For each \(1\leq k \leq K\), our goal is to maximize the terms:
\[\sum_{t = 1}^n p(z_t = k\vert y_{0:n}, \theta^{(0)}) \log p(y_{t} \vert z_{t} = k, \theta).\]
One can recognize here that it is the same linear framework as \eqref{eq:AR-simple} where a specific weight (the posterior probability of being in \(k\) under \(\theta^{(0)}\)) would be assigned to each observation.
In this context, maximizing in \(\theta\) then results in \(K\) weighted linear regressions.

For \(1 \leq k\leq K\), updating
\(\mathbf{B}(k) := \begin{pmatrix} m'(k)\\ \mathbf{A}'(k) \end{pmatrix}\) is done using the weighted least square estimator:
\[\widehat{\mathbf{B}}(k) = \left(\mathbf{X}'\mathbf{W}(k)\mathbf{X} \right)^{-1} \mathbf{X}'\mathbf{W}(k) \mathbf{Y},\]
where
\[\mathbf{W}(k) = \begin{pmatrix}p(z_1 = k\vert y_{0:n}, \theta^{(0)}) & \cdots & 0 \\
\vdots & \ddots & \vdots\\
0 & \cdots & p(z_{n} = k\vert y_{0:n}, \theta^{(0)})
\end{pmatrix}.\]
The estimator of \(\Sigma(k)\)

\hypertarget{linear-dynamics-systems}{%
\chapter{Linear dynamics systems}\label{linear-dynamics-systems}}

In this chapter, we focus on the case where observations consist in a multivariate time series \(y_1, \dots, y_n\) such that for any \(1\leq t \leq n\), \(y_t \in \mathbb{R}^{d_Y}\), we denote:
\[
y_t = 
\begin{pmatrix}
y_{t,1}\\
\vdots\\
y_{t, {d_Y}}
\end{pmatrix}
\]

\hypertarget{model-2}{%
\section{Model}\label{model-2}}

We assume that these observations are realisations of random variables \((Y_1,\dots, Y_n)\) which reflects an \emph{hidden signal} \((X_1, \dots, X_n)\) such that:
\begin{align}
X_1 &\sim \mathcal{N}(\mu_0, \Sigma_0),\nonumber \\
X_t &= b_X + \mathbf{A_X}X_{t -1} + E_{t, X},~2\leq t \leq n \label{eq:Hidden-AR}\\
Y_t &= b_Y + \mathbf{A_Y}X_{t} + E_{t, Y},~1\leq t \leq n \label{eq:Observations-distribution}
\end{align}

where \(\mu_0, b_X \in\mathbb{R}^{d_X}, \mathbf{A_X}\in \mathcal{M}_{d_X\times d_X}, b_Y \in\mathbb{R}^{d_Y}, \mathbf{A_Y}\in \mathcal{M}_{d_Y\times d_Y}\) are parameters, \(E_{t, X}\), (resp. \(E_{t, Y}\)) is a \(d_X\) (resp. \(d_Y\)) dimensionnal vector such that:

\[
E_{t, X} \overset{ind.}{\sim} \mathcal{N}_{d_X}\left(0, \mathbf{\Sigma_X}\right) \left(\text{resp. } E_{t, Y} \overset{ind.}{\sim} \mathcal{N}_{d_Y}\left(0, \mathbf{\Sigma_Y}\right)\right)
\]

  \bibliography{book.bib,packages.bib}

\end{document}
